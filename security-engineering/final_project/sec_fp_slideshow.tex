\documentclass[14pt,compress,dvipsnames,aspectratio=169]{beamer} %usenames,
%\usetheme{Singapore}
\useoutertheme{shadow}
\usetheme{CambridgeUS}
\definecolor{mygreen}{RGB}{150, 255, 210}%186}
\definecolor{leftblue}{RGB}{230,255,255}
\definecolor{rightblue}{RGB}{111,195,223}
\definecolor{lefttron}{RGB}{19,44,65}
\definecolor{myblack}{RGB}{27,27,27}
\definecolor{mypurple}{RGB}{205,87,255}

\usecolortheme{owl}

% \setbeamercolor{section in head/foot}{fg = white,bg=black}
\setbeamercolor{title}{fg=mygreen,bg=black}
\setbeamercolor{titlelike}{fg=yellow,bg=black}
\setbeamercolor{item}{fg=mygreen}
\setbeamercolor{block title}{fg=white,bg=myblack!200}
\setbeamercolor{block body}{bg=normal text.bg!80}
\setbeamertemplate{blocks}[rounded][shadow=true]
\setbeamertemplate{headline}{}
\setbeamertemplate{footline}[frame number]
\setbeamercolor{normal text}{fg=white,bg=myblack}%!89.9}

%Gradient
\setbeamercolor{frametitle}{fg=orange,bg=black}
\setbeamercolor{frametitle right}{fg=white,bg=gray}

\usepackage[utf8]{inputenc}
\usepackage{amsmath}
\usepackage{amsfonts}
\usepackage{amssymb}
\usepackage{graphicx}
\usepackage{shadowtext}
\usepackage{multicol}
\usepackage[makeroom]{cancel}

\usepackage{listings} % For code blocks

%\graphicspath{{./figures/},
%}

%\AtBeginSection{\frame{\sectionpage}}

\usepackage{natbib}
\usepackage{float}
\usepackage{subcaption}
\usepackage{xcolor}
\usepackage{natbib}
\usepackage{bibentry}
\usepackage{animate}
\usepackage{varwidth}
\usepackage{appendixnumberbeamer}

\usepackage{tikz}
\usetikzlibrary{shapes,arrows}

%%%%%%%%% SLIDE 1: TITLE SLIDE %%%%%%%%%
\title{\textbf{Automated Setup of an Air-Gapped Network with a Bastion Host}}
\author{Using Proxmox on a Dell PowerEdge R730}

\date{}  % Get rid of date

\usefonttheme{professionalfonts}

% \usepackage{mydefs}

\setbeamercovered{transparent} 
\setbeamertemplate{navigation symbols}{} 
\titlegraphic{
\begin{center}
\vspace*{-30pt}

\vspace*{10pt}
    \text{by Connor W. (Kolkhis)}
\end{center}
}
%%%%%%%%% END TITLE SLIDE %%%%%%%%%



\begin{document}

\setbeamercovered{invisible}

\begin{frame}[plain]
\titlepage
\end{frame}


%%%% CONTENT FRAMES %%%%

%%%%%%%%%%%%%% FRAME 2: Project Purpose %%%%%%%%%%%%%%%%%%%%
\begin{frame}{Project Purpose}
    Why bastion host + jail?
    \begin{enumerate}
        \item{Improved security. Air-gapped network.} 
        \item{Allow safe ingress.}
        \item{Education. Hands-on experience with an enterprise solution.}
    \end{enumerate}
\end{frame}
%%%%%%%%%%%%%%%%%%%%%%%%%%%%%%%%%%%%%%%%%%%%%%%



%%%%%%%%%%%%%% FRAME 3: Goals %%%%%%%%%%%%%%%%%%%%
\begin{frame}{Goals}
    \begin{enumerate}
        \item{Create a bastion host in the homelab} 
        \item{Use this bastion as an ingress to the rest of the VMs (jumpbox)}
        \item{Route all incoming SSH traffic to the bastion}
    \end{enumerate}
\end{frame}
%%%%%%%%%%%%%%%%%%%%%%%%%%%%%%%%%%%%%%%%%%%%%%%


%%%%%%%%%%%%%% FRAME 4 %%%%%%%%%%%%%%%%%%%%
\begin{frame}{System Diagram}
    \begin{figure}
        \centering
        \includegraphics[width=0.75\linewidth]{bastion_route_diagram.png}
        \caption{SSH Traffic Route}
    \end{figure}
\end{frame}
%%%%%%%%%%%%%%%%%%%%%%%%%%%%%%%%%%%%%%%%%%%%%%%



%--------------------------- The meat ------------------------------%


%%%%%%%%%%%%%% FRAME 5 %%%%%%%%%%%%%%%%%%%%
\begin{frame}{High-Level Overview}
    \begin{enumerate}
        \item{1. Traffic comes into network via SSH} 
        \item{2. Router forwards SSH traffic to bastion host} 
        \item{3. Custom bastion script presents user with destinations}
        \item{4. User enters their destination, they're forwarded there}
    \end{enumerate}
\end{frame}
%%%%%%%%%%%%%%%%%%%%%%%%%%%%%%%%%%%%%%%%%%%%%%%


%%%%%%%%%%%%%% FRAME 6 %%%%%%%%%%%%%%%%%%%%
\begin{frame}{Tech Stack}
    \begin{enumerate}
        \item{\texttt{bash}}
        \item{\texttt{rbash}}
        \item{\texttt{logger} (\texttt{systemd} logging utility)}
        \item{\texttt{chroot}}
        \item{\texttt{ssh}}
            \begin{enumerate}
                \item{SSH \texttt{Match} blocks (conditionals)}
            \end{enumerate}
        \item{Custom Shell - \texttt{bastion.sh}}
    \end{enumerate}
\end{frame}
%%%%%%%%%%%%%%%%%%%%%%%%%%%%%%%%%%%%%%%%%%%%%%%


%%%%%%%%%%%%%% FRAME 7 %%%%%%%%%%%%%%%%%%%%
\begin{frame}{Why use a Chroot Jail?}
    The idea behind a \texttt{chroot} directory is to create an \textbf{isolated
    environment} in which to operate.   

    It forces the user into a directory of our choice, at which point they only have
    access to the tools we give them.  

    There are other ways of directing a user to an isolated environment, such as
    using containers, optionally with namespaces and Cgroups, or other tools.  

    The \texttt{chroot} method is much lower in complexity, easy to set up, and very effective.  
\end{frame}
%%%%%%%%%%%%%%%%%%%%%%%%%%%%%%%%%%%%%%%%%%%%%%%





%%%%%%%%%%%%%% FRAME 8 %%%%%%%%%%%%%%%%%%%%
\begin{frame}{Building the Chroot Directory}
    The \texttt{chroot} directory can be any directory. We're using
    \texttt{/var/chroot} in this particular project.

    We want this directory to pretend to be its own, fully functional, isolated Linux machine.  

    To pretend, it needs the same directory structure as \texttt{/} (root):  
    \begin{enumerate}
        \item{\texttt{/bin}} 
        \item{\texttt{/lib}} 
        \item{\texttt{/lib64}} 
        \item{\texttt{/dev}} 
        \item{\texttt{/etc}} 
        \item{\texttt{/home}} 
        \item{\texttt{/usr/bin}} 
        \item{\texttt{/lib/x86\_64-linux-gnu}} 
    \end{enumerate}
\end{frame}
%%%%%%%%%%%%%%%%%%%%%%%%%%%%%%%%%%%%%%%%%%%%%%%


%%%%%%%%%%%%%% FRAME 9 %%%%%%%%%%%%%%%%%%%%
\begin{frame}{Populating the Jail -- Binaries and Libraries}
    We want the \texttt{chroot} directory to contain \textit{only} the necessary
    binaries to perform its function as an SSH ingress (jumpbox).  

    To that end, we will give it the binaries necessary to perform the jump, and the
    binaries needed to operate the custom shell script.  
    \begin{enumerate}
        \item{\texttt{/usr/bin/rbash} (restricted bash) -- to run the custom script} 
        \item{\texttt{/usr/bin/ping} -- to check connectivity} 
        \item{\texttt{/usr/bin/ssh} -- to perform the jump} 
        \item{\texttt{/usr/bin/logger} -- to perform logging} 
    \end{enumerate}
    Along with the binaries, we must also provide their linked libraries (found via
    the \texttt{ldd} command).  
\end{frame}
%%%%%%%%%%%%%%%%%%%%%%%%%%%%%%%%%%%%%%%%%%%%%%%

%%%%%%%%%%%%%% FRAME 10 %%%%%%%%%%%%%%%%%%%%
\begin{frame}{}
    \begin{enumerate}
        \item{} 
        \item{} 
        \item{} 
        \item{} 
        \item{} 
        \item{} 
    \end{enumerate}
\end{frame}
%%%%%%%%%%%%%%%%%%%%%%%%%%%%%%%%%%%%%%%%%%%%%%%

%%%%%%%%%%%%%% FRAME 11 %%%%%%%%%%%%%%%%%%%% 
\begin{frame}{}
    \begin{enumerate}
        \item{} 
        \item{} 
        \item{} 
        \item{}
    \end{enumerate}
\end{frame}
%%%%%%%%%%%%%%%%%%%%%%%%%%%%%%%%%%%%%%%%%%%%%%%


%%%%%%%%%%%%%% FRAME 12 %%%%%%%%%%%%%%%%%%%% 
\begin{frame}{}
    \begin{enumerate}
        \item{} 
        \item{} 
        \item{} 
        \item{} 
    \end{enumerate}
\end{frame}
%%%%%%%%%%%%%%%%%%%%%%%%%%%%%%%%%%%%%%%%%%%%%%%


%%%%%%%%%%%%%% FRAME 13 %%%%%%%%%%%%%%%%%%%%
\begin{frame}{}
    \begin{enumerate}
        \item{} 
        \item{} 
        \item{} 
    \end{enumerate}
\end{frame}
%%%%%%%%%%%%%%%%%%%%%%%%%%%%%%%%%%%%%%%%%%%%%%%


%%%%%%%%%%%%%% FRAME 14 %%%%%%%%%%%%%%%%%%%%
\begin{frame}{}
    \begin{enumerate}
        \item{} 
        \item{}
        \item{} 
        \item{}
              
    \end{enumerate}
\end{frame}
%%%%%%%%%%%%%%%%%%%%%%%%%%%%%%%%%%%%%%%%%%%%%%%


%%%%%%%%%%%%%% FRAME 15 %%%%%%%%%%%%%%%%%%%%
\begin{frame}{}
    \begin{enumerate}
        \item{}
        \item{}
        \item{}
    \end{enumerate}
\end{frame}
%%%%%%%%%%%%%%%%%%%%%%%%%%%%%%%%%%%%%%%%%%%%%%%





%%%%%%%%%%%%%% FRAME 16 %%%%%%%%%%%%%%%%%%%%
\begin{frame}{}
    \begin{enumerate}
        \item{}
        \item{}
        \item{}
    \end{enumerate}
\end{frame}
%%%%%%%%%%%%%%%%%%%%%%%%%%%%%%%%%%%%%%%%%%%%%%%



%%%%%%%%%%%%%% FRAME 17 %%%%%%%%%%%%%%%%%%%%
\begin{frame}{}
    \begin{enumerate}
        \item{}
        \item{}
        \item{}
        \item{}
    \end{enumerate}
\end{frame}
%%%%%%%%%%%%%%%%%%%%%%%%%%%%%%%%%%%%%%%%%%%%%%%


%%%%%%%%%%%%%% FRAME 18 %%%%%%%%%%%%%%%%%%%%
\begin{}{}
    \begin{enumerate}
        \item{} 
        \item{} 
    \end{enumerate}
\end{frame}
%%%%%%%%%%%%%%%%%%%%%%%%%%%%%%%%%%%%%%%%%%%%%%%


\begin{frame}{References}
    \bibliographystyle{apalike}
    % TODO
    \bibliography{bib}
\end{frame}

\begin{frame}{Questions?}
    \begin{figure}
        \centering
        \includegraphics[width=0.4\linewidth]{kolkhis_github_qr.png}
        \caption{Kolkhis on GitHub}
        \label{}
    \end{figure}
\end{frame}
\end{document}
