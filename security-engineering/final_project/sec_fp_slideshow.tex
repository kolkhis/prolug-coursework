\documentclass[14pt,compress,dvipsnames,aspectratio=169]{beamer} %usenames,
\useoutertheme{shadow}
\usetheme{CambridgeUS}
\definecolor{mygreen}{RGB}{150, 255, 210}%186}
\definecolor{leftblue}{RGB}{230,255,255}
\definecolor{rightblue}{RGB}{111,195,223}
\definecolor{lefttron}{RGB}{19,44,65}
\definecolor{myblack}{RGB}{27,27,27}
\definecolor{mypurple}{RGB}{205,87,255}

\usecolortheme{owl}

% \setbeamercolor{section in head/foot}{fg = white,bg=black}
\setbeamercolor{title}{fg=mygreen,bg=black}
\setbeamercolor{titlelike}{fg=yellow,bg=black}
\setbeamercolor{item}{fg=mygreen}
\setbeamercolor{block title}{fg=white,bg=myblack!200}
\setbeamercolor{block body}{bg=normal text.bg!80}
\setbeamertemplate{blocks}[rounded][shadow=true]
\setbeamertemplate{headline}{}
\setbeamertemplate{footline}[frame number]
\setbeamercolor{normal text}{fg=white,bg=myblack}%!89.9}

%Gradient
\setbeamercolor{frametitle}{fg=orange,bg=black}
\setbeamercolor{frametitle right}{fg=white,bg=gray}

\usepackage[utf8]{inputenc}
\usepackage{amsmath}
\usepackage{amsfonts}
\usepackage{amssymb}
\usepackage{graphicx}
\usepackage{shadowtext}
\usepackage{multicol}
\usepackage[makeroom]{cancel}

\usepackage{listings} % For code blocks

%\graphicspath{{./figures/},
%}

%\AtBeginSection{\frame{\sectionpage}}

\usepackage{natbib}
\usepackage{float}
\usepackage{subcaption}
\usepackage{xcolor}
\usepackage{natbib}
\usepackage{bibentry}
\usepackage{animate}
\usepackage{varwidth}
\usepackage{appendixnumberbeamer}

\usepackage{tikz}
\usetikzlibrary{shapes,arrows}

%%%%%%%%% SLIDE 1: TITLE SLIDE %%%%%%%%%
\title{\textbf{Automated Setup of an Air-Gapped Network with a Bastion Host}}
\author{Using Bash}

\date{}  % Get rid of date

\usefonttheme{professionalfonts}

% \usepackage{mydefs}

\setbeamercovered{transparent} 
\setbeamertemplate{navigation symbols}{} 
\titlegraphic{
\begin{center}
\vspace*{-30pt}

\vspace*{10pt}
    \text{by Connor W. (Kolkhis)}
\end{center}
}
%%%%%%%%% END TITLE SLIDE %%%%%%%%%



\begin{document}

\setbeamercovered{invisible}

\begin{frame}[plain]
\titlepage
\end{frame}


%%%% CONTENT FRAMES %%%%



%%%%%%%%%%%%%% FRAME 2: Goals %%%%%%%%%%%%%%%%%%%%
\begin{frame}{Goals}
    \begin{itemize}
        % \item{Create a bastion host in the homelab} 
        % \item{Use this bastion as an ingress to the rest of the VMs (jumpbox)}
        % \item{Route all incoming SSH traffic to the bastion}
        \item{\textbf{Build a Bastion
            Host} -- Gateway for SSH ingress}

        \item{\textbf{Restrict User
            Access} -- Use \texttt{chroot} and \texttt{rbash} to isolate the shell
            environment}

        \item{\textbf{Route Ingress SSH
            Traffic} -- Use \texttt{sshd} \texttt{Match} blocks to force logins through the jail}

        \item{\textbf{Offer Controlled
            Forwarding}  -- Custom script presents user with allowed destinations}
    \end{itemize}
\end{frame}
%%%%%%%%%%%%%%%%%%%%%%%%%%%%%%%%%%%%%%%%%%%%%%%

%%%%%%%%%%%%%% FRAME 3: Project Purpose %%%%%%%%%%%%%%%%%%%%
\begin{frame}{Project Purpose}
    \begin{itemize}
        \item{\textbf{Secure gateway} -- Centralized and controlled access to internal systems} 
            \vspace{0.5cm}
        \item{\textbf{Air-Gapped design} -- Ensure no direct ingress to internal hosts}
            \vspace{0.5cm}
        \item{\textbf{Hands-on learning} -- Simulate real-world security architecture}
    \end{itemize}
\end{frame}
%%%%%%%%%%%%%%%%%%%%%%%%%%%%%%%%%%%%%%%%%%%%%%%



%%%%%%%%%%%%%% FRAME 4 %%%%%%%%%%%%%%%%%%%%
\begin{frame}{System Diagram}
    \begin{figure}
        \centering
        \includegraphics[width=0.75\linewidth]{bastion_route_diagram.png}
        \caption{SSH Traffic Route}
    \end{figure}
\end{frame}
%%%%%%%%%%%%%%%%%%%%%%%%%%%%%%%%%%%%%%%%%%%%%%%



%--------------------------- The meat ------------------------------%


%%%%%%%%%%%%%% FRAME 5 %%%%%%%%%%%%%%%%%%%%
\begin{frame}{High-Level Overview}
        1. Traffic comes into network via SSH 
        \vspace{1.15cm}

        2. Router forwards SSH traffic to bastion host
        \vspace{1.15cm}

        3. Custom bastion script presents user with destinations
        \vspace{1.15cm}

        4. User enters their destination, they're forwarded there
\end{frame}
%%%%%%%%%%%%%%%%%%%%%%%%%%%%%%%%%%%%%%%%%%%%%%%


%%%%%%%%%%%%%% FRAME 6 %%%%%%%%%%%%%%%%%%%%
\begin{frame}{Tech Stack / Tools Used}
    \begin{itemize}
        \item{\texttt{Linux}}
        \item{\texttt{bash}}
        \item{\texttt{rbash}}
        \item{\texttt{logger} (\texttt{systemd} logging utility)}
        \item{\texttt{chroot}}
        \item{\texttt{ping}}
        \item{\texttt{ssh}}
            \begin{enumerate}
                \item{SSH \texttt{Match} blocks (conditionals)}
            \end{enumerate}
        \item{Custom Shell -- \texttt{bastion.sh}}
    \end{itemize}
\end{frame}
%%%%%%%%%%%%%%%%%%%%%%%%%%%%%%%%%%%%%%%%%%%%%%%


%%%%%%%%%%%%%% FRAME 7 %%%%%%%%%%%%%%%%%%%%
\begin{frame}{Why use a Chroot Jail?}
    The idea behind a \texttt{chroot} directory is to create an \textbf{isolated
    environment} in which to send users.   

    \vspace{0.25cm}
    It forces the user into a sandbox and can only access the tools we give them.  

    \vspace{0.25cm}
    There are other ways of directing a user to an isolated environment.  

    \vspace{0.25cm}
    The \texttt{chroot} method is much lower in complexity, easy to set up, and very effective.  
    \begin{itemize}
        \item Simple to configure
        \item No extra dependencies
        \item Easily audited (single log stream)
    \end{itemize}
\end{frame}
%%%%%%%%%%%%%%%%%%%%%%%%%%%%%%%%%%%%%%%%%%%%%%%





%%%%%%%%%%%%%% FRAME 8 %%%%%%%%%%%%%%%%%%%%
\begin{frame}{Building the Chroot Directory}
    The \texttt{chroot} directory can be any directory. We're using
    \texttt{/var/chroot} in this particular project.

    We want this directory to pretend to be its own standalone Linux machine, so we
    need to emulate the root directory structure:  

    \begin{itemize}
        \item{\texttt{/bin}} 
        \item{\texttt{/lib}} 
        \item{\texttt{/lib64}} 
        \item{\texttt{/dev}} 
        \item{\texttt{/etc}} 
        \item{\texttt{/home}} 
        \item{\texttt{/usr/bin}} 
        \item{\texttt{/lib/x86\_64-linux-gnu}} 
    \end{itemize}
\end{frame}
%%%%%%%%%%%%%%%%%%%%%%%%%%%%%%%%%%%%%%%%%%%%%%%


%%%%%%%%%%%%%% FRAME 9 %%%%%%%%%%%%%%%%%%%%
\begin{frame}{Populating the Jail -- Binaries and Libraries}
    We want the \texttt{chroot} directory to contain \textit{only} the necessary
    binaries to perform its function as an SSH ingress (jumpbox).  

    To that end, we will give it the binaries necessary to perform the jump, and the
    binaries needed to operate the custom shell script.  
    \begin{itemize}
        \item{\texttt{/usr/bin/rbash} (restricted bash) -- to run the custom script} 
        \item{\texttt{/usr/bin/ping} -- to check connectivity} 
        \item{\texttt{/usr/bin/ssh} -- to perform the jump} 
        \item{\texttt{/usr/bin/logger} -- to perform logging} 
    \end{itemize}
    Along with the binaries, we must also provide their linked libraries (found via
    the \texttt{ldd} command).  
\end{frame}
%%%%%%%%%%%%%%%%%%%%%%%%%%%%%%%%%%%%%%%%%%%%%%%


%%%%%%%%%%%%%% FRAME 10 %%%%%%%%%%%%%%%%%%%%
\begin{frame}{Populating the Jail -- Special Files in \texttt{/dev}}
    Many Linux programs require \textbf{character special files} in order to function
    properly.

    In particular, these special files are needed in the \texttt{/var/chroot/dev} dir:
    \begin{itemize}
        \item{\texttt{/dev/null} -- Used by almost every Unix tool for redirection} 
        \item{\texttt{/dev/zero} -- Used for file generation and memory allocation} 
        \item{\texttt{/dev/random} and \texttt{/dev/urandom} -- Used by \texttt{ssh} to generate entropy } 
        \item{\texttt{/dev/tty} -- Needed for user input/interaction} 
    \end{itemize}
    \vspace{0.5em}
    Without these special files, \textbf{some or all} of the binaries in the jail may
    fail or act strangely.
\end{frame}
%%%%%%%%%%%%%%%%%%%%%%%%%%%%%%%%%%%%%%%%%%%%%%%

%%%%%%%%%%%%%% FRAME 11 %%%%%%%%%%%%%%%%%%%%
\begin{frame}{Populating the Jail -- System Files}
    To pretend it's a whole Linux environment, the jail needs key system files:
    \begin{itemize}
        \item{\texttt{/etc/passwd} -- Required by \texttt{rbash/ssh} to for checking the user's UID and shell} 
        \item{\texttt{/etc/group} -- Allows checking the user's groups, affects permissions} 
        \item{\texttt{/etc/nsswitch.conf} -- Needed for name resolution to work
            properly (e.g., DNS)} 
        \item{\texttt{/etc/hosts} -- For local name resolution when
            \texttt{nsswitch.conf} includes \texttt{hosts: files dns}} 
        \item{\texttt{/lib/x86\_64-linux-gnu/*nss*} -- Name Switch Service libraries
            used by \texttt{glibc}, required for name resolution} 
    \end{itemize}
    \vspace{0.5em}
\end{frame}
%%%%%%%%%%%%%%%%%%%%%%%%%%%%%%%%%%%%%%%%%%%%%%%


%%%%%%%%%%%%%% FRAME 12 %%%%%%%%%%%%%%%%%%%%
\begin{frame}{SSH \texttt{Match} Rule}
    The jailed user has its own user account on the system (\texttt{juvie}).  

    \vspace{0.25cm}
    We can put an entry in \texttt{/etc/ssh/sshd\_config} to \textbf{match} logins
    as this user.  
    \vspace{0.25cm}

    \texttt{Match User juvie}

    \hspace{1.0cm}\texttt{ChrootDirectory /var/chroot}

    \hspace{1.0cm}\texttt{PasswordAuthentication yes}

    \hspace{1.0cm}\texttt{AuthenticationMethods password}

    \vspace{0.25cm}
    This entry:
    \begin{itemize}
        \item{Matches SSH logins as the \texttt{User} named \texttt{juvie}} 
        \item{Sets their \texttt{ChrootDirectory} to \texttt{/var/chroot}} 
        \item{Allows password login for this user} 
    \end{itemize}
\end{frame}
%%%%%%%%%%%%%%%%%%%%%%%%%%%%%%%%%%%%%%%%%%%%%%%

%%%%%%%%%%%%%% FRAME 13 %%%%%%%%%%%%%%%%%%%% 
\begin{frame}{Custom Shell -- \texttt{bastion.sh} with \texttt{rbash}}
    \begin{figure}
        \centering
        \includegraphics[width=0.5\linewidth]{juvie-etc_passwd-entry.png}
    \end{figure}
    The user does not have access to a normal shell (\texttt{bash} or \texttt{sh}).
    \vspace{0.25cm}

    They are only given access to a custom shell script called \texttt{bastion.sh},
    which uses \texttt{rbash} instead of \texttt{bash}.
    \vspace{0.25cm}

    This script presents the user with options for servers to jump to (parsed from an
    SSH config file).  
    \vspace{0.25cm}

    If the user enters an option that is not presented to them, they are kicked out.  
    \vspace{0.25cm}

    The \texttt{bastion.sh} script uses \texttt{logger} to log user actions to
    \texttt{journald}.  
    % The script runs on \texttt{rbash} 
\end{frame}
%%%%%%%%%%%%%%%%%%%%%%%%%%%%%%%%%%%%%%%%%%%%%%%


%%%%%%%%%%%%%% FRAME 14 %%%%%%%%%%%%%%%%%%%% 
\begin{frame}{Why \texttt{rbash} instead of \texttt{bash}?}
    \texttt{rbash} (Restricted Bash) is a safer variant of \texttt{bash}.  

    It disables dangerous features that could let a user escape the jail.  
    \vspace{0.25cm}

    \texttt{rbash} \textbf{disallows} these actions:
    \vspace{0.25cm}
    \begin{itemize}
        \item{Using \texttt{cd} to escape the jail} 
        \item{Setting or unsetting \texttt{PATH}, \texttt{SHELL}, \texttt{HISTFILE}, \texttt{ENV}/\texttt{BASH\_ENV} }
        \item{Output redirection (e.g., \texttt{>}, \texttt{>\&}, \texttt{>>}, etc.)} 
        \item{Running commands with slashes (e.g., \texttt{/bin/sh})} 
        \item{Using \texttt{exec} to replace the shell process} 
        \item{and much more...} 
    \end{itemize}
    \vspace{0.15cm}
    This reduces the attack surface of the custom shell.  
\end{frame}
%%%%%%%%%%%%%%%%%%%%%%%%%%%%%%%%%%%%%%%%%%%%%%%

%%%%%%%%%%%%%% FRAME 15 %%%%%%%%%%%%%%%%%%%%
\begin{frame}{Scripting Automation}
    The whole setup process has been automated using three Bash scripts.  
    \vspace{1.0cm}
    \begin{itemize}
        \item{\texttt{setup-chroot-jail} -- Sets everything up} 
        \vspace{1.0cm}
        \item{\texttt{generate-destinations} -- Helper script that parses an SSH
            config file into a more script-friendly format}
        \vspace{1.0cm}
        \item{\texttt{reset-chroot-jail} -- Helper script used during development to
            remove the chroot jail and jailed user}
        % \item{\texttt{}}
    \end{itemize}
\end{frame}
%%%%%%%%%%%%%%%%%%%%%%%%%%%%%%%%%%%%%%%%%%%%%%%

%%%%%%%%%%%%%% FRAME 16 %%%%%%%%%%%%%%%%%%%%
\begin{frame}{Future Improvements}
    \begin{itemize}
        \item{Enable \texttt{fail2ban} for bastion host} 
        \item{Parse an ansible inventory file for SSH destinations} 
        \item{Use \texttt{readonly bind mounts} instead of copying binaries/libraries}
        \item{Implement per-user logging and session auditing (e.g., per-IP)}
        \item{Add MFA or TOTP-based verification on top of password login}
        \item{Add AppArmor or Seccomp profile to further restrict jailed shell behavior}
        \item{Replace \texttt{rbash} with a minimal statically compiled Go binary as a shell}
    \end{itemize}
\end{frame}
%%%%%%%%%%%%%%%%%%%%%%%%%%%%%%%%%%%%%%%%%%%%%%%

%%%%%%%%%%%%%% FRAME 17 %%%%%%%%%%%%%%%%%%%%
\begin{frame}{Demo}
    (show them how it works)
\end{frame}
%%%%%%%%%%%%%%%%%%%%%%%%%%%%%%%%%%%%%%%%%%%%%%%

%%%%%%%%%%%%%% FRAME 18 %%%%%%%%%%%%%%%%%%%%
\begin{frame}{Questions?}
    \begin{itemize}
        \vspace{1.0cm}
        \item{Ask me right now}
        \vspace{1.0cm}
        \item{@ me on Discord} 
    \end{itemize}
\end{frame}
%%%%%%%%%%%%%%%%%%%%%%%%%%%%%%%%%%%%%%%%%%%%%%%

\begin{frame}{References}
    \begin{itemize}
        \item{https://killercoda.com/het-tanis/course/Linux-Labs/204-building-a-chroot-jail}
        \item{https://killercoda.com/het-tanis/course/Linux-Labs/210-building-a-bastion-host}
        \item{\texttt{man chroot}}
        \item{\texttt{man logger}}
        \item{\texttt{man rbash}}
        \item{And, as always, \texttt{man bash}}
    \end{itemize}
\end{frame}

\begin{frame}{Questions?}
    \begin{figure}
        \centering
        \includegraphics[width=0.4\linewidth]{kolkhis_github_qr.png}
        \caption{Kolkhis on GitHub}
        \label{GH-QR-Code}
    \end{figure}
\end{frame}

\end{document}
