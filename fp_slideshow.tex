\documentclass[14pt,compress,usenames,dvipsnames,aspectratio=169]{beamer}
%\usetheme{Singapore}
\useoutertheme{shadow}
\usetheme{CambridgeUS}
\definecolor{mygreen}{RGB}{150, 255, 210}%186}
\definecolor{leftblue}{RGB}{230,255,255}
\definecolor{rightblue}{RGB}{111,195,223}
\definecolor{lefttron}{RGB}{19,44,65}
\definecolor{myblack}{RGB}{27,27,27}
\definecolor{mypurple}{RGB}{205,87,255}

\usecolortheme{owl}

% \setbeamercolor{section in head/foot}{fg = white,bg=black}
\setbeamercolor{title}{fg=mygreen,bg=black}
\setbeamercolor{titlelike}{fg=yellow,bg=black}
\setbeamercolor{item}{fg=mygreen}
\setbeamercolor{block title}{fg=white,bg=myblack!200}
\setbeamercolor{block body}{bg=normal text.bg!80}
\setbeamertemplate{blocks}[rounded][shadow=true]
\setbeamertemplate{headline}{}
\setbeamertemplate{footline}[frame number]
\setbeamercolor{normal text}{fg=white,bg=myblack}%!89.9}

%Gradient
\setbeamercolor{frametitle}{fg=orange,bg=black}
\setbeamercolor{frametitle right}{fg=white,bg=gray}

\usepackage[utf8]{inputenc}
\usepackage{amsmath}
\usepackage{amsfonts}
\usepackage{amssymb}
\usepackage{graphicx}
\usepackage{shadowtext}
\usepackage{multicol}
\usepackage[makeroom]{cancel}

\usepackage{listings} % For code blocks

%\graphicspath{{./figures/},
%}

%\AtBeginSection{\frame{\sectionpage}}

\usepackage{natbib}
\usepackage{float}
\usepackage{subcaption}
\usepackage{xcolor}
\usepackage{natbib}
\usepackage{bibentry}
\usepackage{animate}
\usepackage{varwidth}
\usepackage{appendixnumberbeamer}

\usepackage{tikz}
\usetikzlibrary{shapes,arrows}

%%%%%%%%% TITLE SLIDE %%%%%%%%%
\title{\textbf{Automated Setup of a Monitoring Stack using Ansible}}
\author{Using Proxmox on a Dell PowerEdge R730}

\date{}  % Get rid of date

\usefonttheme{professionalfonts}

\usepackage{mydefs}

\setbeamercovered{transparent} 
\setbeamertemplate{navigation symbols}{} 
\titlegraphic{
\begin{center}
\vspace*{-30pt}

\vspace*{10pt}
    \text{by Kolkhis}
\end{center}
}
%%%%%%%%% END TITLE SLIDE %%%%%%%%%



\begin{document}

\setbeamercovered{invisible}

\begin{frame}[plain]
\titlepage
\end{frame}

%%%% CONTENT FRAMES %%%%

%%%%%%%%%%%%%% FRAME 1 %%%%%%%%%%%%%%%%%%%%
% FRAME ONE: GOAL
\begin{frame}{Goals}
    \begin{enumerate}
        \item{Set up a Type 1 hypervisor for running Virtual Machines and
            Containerized Applications.} 
        \item{Set up monitoring tools for analyzing resource usage and logs. }
        \item{Automate the setup of monitoring tools on all VMs and Containers using
            Ansible.}
    \end{enumerate}
\end{frame}
%%%%%%%%%%%%%%%%%%%%%%%%%%%%%%%%%%%%%%%%%%%%%%%


%%%%%%%%%%%%%%% FRAME 2 %%%%%%%%%%%%%%%%%%%%
%% FRAME: Machine Setup
%\begin{frame}{Getting the Machine Ready}
%    \begin{enumerate}
%        \item{I discovered early on that disks in this type of server won't be recognized by installers by default.} 
%        \item{The server uses a Hardware RAID controller, so all disks are RAID devices by default.}
%        \item{I opted to use the BIOS/UEFI convert my SSDs to Non-RAID Disks instead of messing with the Hardware RAID Controller.}
%        \item{This took me a long time to figure out.}
%    \end{enumerate}
%    \begin{figure}
%        \centering
%        \includegraphics[width=0.5\linewidth]{installer-no-disks-error.png}  

%        \caption{Proxmox installer error: No supported disks.}
%        \label{fig:Proxmox installer error - no disks.}
%    \end{figure}
%\end{frame}
%%%%%%%%%%%%%%%%%%%%%%%%%%%%%%%%%%%%%%%%%%%%%%%%


%%%%%%%%%%%%%%% FRAME 3 %%%%%%%%%%%%%%%%%%%%
%\begin{frame}{Installing Proxmox}
%    \begin{enumerate}
%        \item{I installed "Proxmox VE (Terminal UI)", as I didn't see the need for a
%        GUI.} 
%        \item{Everything can be done from the Proxmox Web UI, from configuring
%        storage to launching VMs and containers.}
%        \item{I had quite a bit of trouble with the initial installation. I'd never
%            touched an enterprise server before.}
%    \end{enumerate}
%\end{frame}
%%%%%%%%%%%%%%%%%%%%%%%%%%%%%%%%%%%%%%%%%%%%%%%%

%%%%%%%%%%%%%%% FRAME 4 (Picture) %%%%%%%%%%%%%%%%%%%%
%\begin{frame}{}
%    \begin{figure}
%        \centering
%        \includegraphics[width=0.5\linewidth]{pve_setup.jpeg}
%        \caption{Photo of the server during setup}
%        \label{fig:setup-pic}
%    \end{figure}
%\end{frame}
%%%%%%%%%%%%%%%%%%%%%%%%%%%%%%%%%%%%%%%%%%%%%%%%

%%%%%%%%%%%%%%% FRAME: Deciding on storage %%%%%%%%%%%%%%%%%%%%
%\begin{frame}{Deciding on the Type of Storage to Use}
%    I had a hard time deciding what type of storage to use. 
%    \begin{enumerate}
%        \item{I initially started using LVM. But this did not support snapshots or
%            data sharing.}
%        \item{I tried making partitions using `ext4` type filesystems, but found they didn't support VM snapshots either.}
%        \item{I looked into Ceph, since it has native support in Proxmox, but setup
%            looked a little too complex for the timeline of this project.}
%        \item{I decided to use ZFS. Also has native support, with feature support 
%            (snapshots/sharing). I just wanted to get it up and running.}
%    \end{enumerate}
%\end{frame}
%%%%%%%%%%%%%%%%%%%%%%%%%%%%%%%%%%%%%%%%%%%%%%%%

%%%%%%%%%%%%%%% FRAME: USING ZFS %%%%%%%%%%%%%%%%%%%%
%\begin{frame}{Creating ZFS for Storage}
%    I wiped the two disks I was using for `ext4` filesystems and converted them
%    to a `zfs` pool.  

%    Then I used the Proxmox Web UI to provision the `zfs` pool for virtual
%    machines.  
 
%\verb+

%wipefs -a /dev/sdb /dev/sdc;  
 
%zpool create vmdata /dev/sdb/ /dev/sdc;  

 
%\+
%\end{frame}
%%%%%%%%%%%%%%%%%%%%%%%%%%%%%%%%%%%%%%%%%%%%%%%%

%%%%%%%%%%%%%%%%%%%%%%%%%%%% MONITORING %%%%%%%%%%%%%%%%%%%%%%%%%%%%%%%
%%%%%%%%%%%%%% FRAME Central VM Setup %%%%%%%%%%%%%%%%%%%%
\begin{frame}{Monitoring, Pt1: Create a Centralized VM}
    \begin{enumerate}
        \item{Creating a centralized VM to host the monitoring tools.} 
        \item{All other nodes will send metrics and logs to this VM.}
        \item{I gave this VM 2 CPUs, 6GB RAM, and 80GB disk space.}
    \end{enumerate}
    \begin{figure}
        \centering
        \includegraphics[width=0.75\linewidth]{monitoring_stack_viz.png}
        \caption{Monitoring Schema}
        \label{fig:enter-label}
    \end{figure} 
\end{frame}
%%%%%%%%%%%%%%%%%%%%%%%%%%%%%%%%%%%%%%%%%%%%%%%


%%%%%%%%%%%%%% FRAME Write Playbooks to Install Tools on Central VM %%%%%%%%%%%%%%%%%%%%
\begin{frame}{Monitoring, Pt2: Install the Tools on Central VM}
    Writing an Ansible playbook to install tools on the central VM.
    \begin{enumerate}
        \item{I started with the central VM to host these tools:} 
            \begin{enumerate}
                \item{Prometheus (Metrics TSDB)}
                \item{Grafana (Visualizations)}
                \item{Loki (Log Storage)}
            \end{enumerate}
        \item{I wrote a series of playbooks that would just run on that VM and install these tools.}
    \end{enumerate}
\end{frame}
%%%%%%%%%%%%%%%%%%%%%%%%%%%%%%%%%%%%%%%%%%%%%%%


%%%%%%%%%%%%%% FRAME Grafana Playbook %%%%%%%%%%%%%%%%%%%%
\begin{frame}{Grafana Setup Playbook}
The Grafana plabook was simple. It can be installed with most package managers.  

    Tasks:
    \begin{enumerate}
        \item{Pull the Grafana GPG key.} 
        \item{Add the Grafana repository, signed by the GPG key.}
        \item{Install using the host OS's package manager.}
        \item{Add a handler that starts/enables the Grafana-server systemd service.}
        \item{Use blocks with conditions to detect the OS family to determine where
            to store the GPG key and which package manager to use.}
            %whether to use either dnf/yum or apt.}
        \item{Expose port 3000 if \verb+firewalld\+ is running.}
    \end{enumerate}
\end{frame}
%%%%%%%%%%%%%%%%%%%%%%%%%%%%%%%%%%%%%%%%%%%%%%%


%%%%%%%%%%%%%% FRAME Provisioning Datasources && dashboards %%%%%%%%%%%%%%%%%%%%
\begin{frame}{Provisioning Data Sources and Dashboards}
    \begin{enumerate}
        \item{Default dashboards and datasources can be stored in json format.}
        \item{Configure the data sources directly in yaml format in
            \verb+/etc/grafana/provisioning/datasources/\+}
        \item{Configure the dashboards in json format, using a yaml config to 
                specify the path to the dashboard files in 
                \verb+/etc/grafana/provisioning/datasources/\+}
    \end{enumerate}
\end{frame}
%%%%%%%%%%%%%%%%%%%%%%%%%%%%%%%%%%%%%%%%%%%%%%%
% TODO: Add example config for dashboards and data sources

% Data Source Provisioning File
\begin{figure}
    \centering
    \includegraphics[width=0.55\linewidth]{grafana_datasource_provisioning_file.png}
    \caption{Grafana Datasource Provisioning File}
    \label{fig:enter-label}
\end{figure}

% Dashboard Provisioning File
\begin{figure}
    \centering
    \includegraphics[width=0.55\linewidth]{grafana_dashboard_provisioning_example.png}
    \caption{Grafana Dashboard Provisioning File}
    \label{fig:enter-label}
\end{figure}

%%%%%%%%%%%%%% FRAME Prometheus Playbook %%%%%%%%%%%%%%%%%%%%
\begin{frame}{Prometheus Setup Playbook}
    Tasks:
    \begin{enumerate}
        \item{Pull the tarball from Github.} 
        \item{Create necessary directories and user account.} 
        \item{Extract tarball into \verb+/var/lib/prometheus\+} 
        \item{Copy the binary and service files to their respective directories.} 
        \item{Create (or copy) some default configuration files.} 
        \item{Expose port 9090 if accessing externally.} 
    \end{enumerate}
\end{frame}
%%%%%%%%%%%%%%%%%%%%%%%%%%%%%%%%%%%%%%%%%%%%%%%

%%%%%%%%%%%%%% FRAME Prometheus File-based Service Discovery %%%%%%%%%%%%%%%%%%%% 
\begin{frame}{Prometheus Service Discovery}
    Prometheus supports file-based service discovery.  
    To set it up:
    \begin{enumerate}
        \item{Create a default \verb+targets.json\+} 
        \item{Add a file\_sd\_configs entry in \verb+/etc/prometheus/prometheus.yml\+
            (Prometheus configuration file)} 
        \item{Point the file\_sd\_configs to your targets.json file.} 
        \item{Whenever a change is made in targets.json, Prometheus will detect the
            change and update accordingly.}
    \end{enumerate}
\end{frame}
%%%%%%%%%%%%%%%%%%%%%%%%%%%%%%%%%%%%%%%%%%%%%%%




%%%%%%%%%%%%%% FRAME Node Exporter Playbook %%%%%%%%%%%%%%%%%%%% 
\begin{frame}{Node Exporter Playbook}
    This process was similar to that of Prometheus.  

    Tasks:
    \begin{enumerate}
        \item{Create user account for \verb+node\_exporter.\+} 
        \item{Download and extract the node\_exporter tarball.} 
        \item{Clone the default config\_files repository for the service files.} 
        \item{Copy binary to \verb+/usr/sbin\+} 
        \item{Copy service files to \verb+/etc/systemd/system/\+} 
        \item{Read targets.json from the Prometheus node and append the new target.} 
        \item{If firewalld is running, expose port 9100. } 
        \item{Use systemctl to start/enable node\_exporter.} 
    \end{enumerate}
\end{frame}
%%%%%%%%%%%%%%%%%%%%%%%%%%%%%%%%%%%%%%%%%%%%%%%


%%%%%%%%%%%%%% FRAME Node Exporter Targets %%%%%%%%%%%%%%%%%%%%
\begin{frame}{Node Exporter Playbook, Pt2}
    Benefits of using this method:
    \begin{enumerate}
        \item{Every time this playbook deploys node\_exporter on a host, it adds 
            that host to targets.json (/etc/prometheus/target.json).} 
        \item{Prometheus will detect the change in this file and update.} 
        \item{This allows for dynamic updates of Prometheus scrape targets without
            having to restart the Prometheus service.} 
    \end{enumerate}
\end{frame}
%%%%%%%%%%%%%%%%%%%%%%%%%%%%%%%%%%%%%%%%%%%%%%%



%%%%%%%%%%%%%% EMPTY FRAME %%%%%%%%%%%%%%%%%%%%
\begin{frame}{}
    \begin{enumerate}
        \item{} 
        \item{} 
        \item{} 
    \end{enumerate}
\end{frame}
%%%%%%%%%%%%%%%%%%%%%%%%%%%%%%%%%%%%%%%%%%%%%%%


\begin{frame}{References}
    \bibliographystyle{apalike}
        Proxmox docs: https://pve.proxmox.com/wiki/  
        Ansible docs: https://docs.ansible.com/ansible/latest/collections/ansible/builtin  
        ProLUG Labs on Killercoda: https://killercoda.com/het-tanis/course/Linux-Labs  
        Prometheus File-based Service Discovery Config https://prometheus.io/docs/prometheus/latest/configuration/configuration/#file_sd_config  
    \bibliography{bib}
\end{frame}

\end{document}
