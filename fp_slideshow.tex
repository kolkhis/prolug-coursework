\documentclass[14pt,compress,usenames,dvipsnames,aspectratio=169]{beamer}
%\usetheme{Singapore}
\useoutertheme{shadow}
\usetheme{CambridgeUS}
\definecolor{mygreen}{RGB}{150, 255, 210}%186}
\definecolor{leftblue}{RGB}{230,255,255}
\definecolor{rightblue}{RGB}{111,195,223}
\definecolor{lefttron}{RGB}{19,44,65}
\definecolor{myblack}{RGB}{27,27,27}
\definecolor{mypurple}{RGB}{205,87,255}

\usecolortheme{owl}

% \setbeamercolor{section in head/foot}{fg = white,bg=black}
\setbeamercolor{title}{fg=mygreen,bg=black}
\setbeamercolor{titlelike}{fg=yellow,bg=black}
\setbeamercolor{item}{fg=mygreen}
\setbeamercolor{block title}{fg=white,bg=myblack!200}
\setbeamercolor{block body}{bg=normal text.bg!80}
\setbeamertemplate{blocks}[rounded][shadow=true]
\setbeamertemplate{headline}{}
\setbeamertemplate{footline}[frame number]
\setbeamercolor{normal text}{fg=white,bg=myblack}%!89.9}

%Gradient
\setbeamercolor{frametitle}{fg=orange,bg=black}
\setbeamercolor{frametitle right}{fg=white,bg=gray}

\usepackage[utf8]{inputenc}
\usepackage{amsmath}
\usepackage{amsfonts}
\usepackage{amssymb}
\usepackage{graphicx}
\usepackage{shadowtext}
\usepackage{multicol}
\usepackage[makeroom]{cancel}

\usepackage{listings} % For code blocks

%\graphicspath{{./figures/},
%}

%\AtBeginSection{\frame{\sectionpage}}

\usepackage{natbib}
\usepackage{float}
\usepackage{subcaption}
\usepackage{xcolor}
\usepackage{natbib}
\usepackage{bibentry}
\usepackage{animate}
\usepackage{varwidth}
\usepackage{appendixnumberbeamer}

\usepackage{tikz}
\usetikzlibrary{shapes,arrows}

%%%%%%%%% TITLE SLIDE %%%%%%%%%
\title{\textbf{Setting up a Proxmox Homelab on an Enterprise Server}}
\author{Using a Dell PowerEdge R730}

\date{}  % Get rid of date

\usefonttheme{professionalfonts}

\usepackage{mydefs}

\setbeamercovered{transparent} 
\setbeamertemplate{navigation symbols}{} 
\titlegraphic{
\begin{center}
\vspace*{-30pt}

\vspace*{10pt}
    \text{by Kolkhis}
\end{center}
}
%%%%%%%%% END TITLE SLIDE %%%%%%%%%



\begin{document}

\setbeamercovered{invisible}

\begin{frame}[plain]
\titlepage
\end{frame}

%%%% CONTENT FRAMES %%%%

%%%%%%%%%%%%%% FRAME 1 %%%%%%%%%%%%%%%%%%%%
% FRAME ONE: GOAL
\begin{frame}{Goals}
    \begin{enumerate}
        \item{Set up a Type 1 hypervisor for running Virtual Machines and
            Containerized Applications.} 
        \item{Set up monitoring tools for analyzing resource usage and logs. }
        \item{Automate the setup of monitoring tools on all VMs and Containers using
            Ansible.}
    \end{enumerate}
\end{frame}
%%%%%%%%%%%%%%%%%%%%%%%%%%%%%%%%%%%%%%%%%%%%%%%


%%%%%%%%%%%%%% FRAME 2 %%%%%%%%%%%%%%%%%%%%
% FRAME: Machine Setup
\begin{frame}{Getting the Machine Ready}
    \begin{enumerate}
        \item{I discovered early on that disks in this type of server won't be recognized by installers by default.} 
        \item{The server uses a Hardware RAID controller, so all disks are RAID devices by default.}
        \item{I opted to use the BIOS/UEFI convert my SSDs to Non-RAID Disks instead of messing with the Hardware RAID Controller.}
        \item{This took me a long time to figure out.}
        \item{Later, `smartctl` showed that the 800GB drive had 7 years of active use.} 
        \item{I wound up replacing the 800GB drive with two 512GB SSDs.} 
    \end{enumerate}
\end{frame}
%%%%%%%%%%%%%%%%%%%%%%%%%%%%%%%%%%%%%%%%%%%%%%%

%%%%%%%%%%%%%% FRAME 3 %%%%%%%%%%%%%%%%%%%%
\begin{frame}{Installing Proxmox}
    \begin{enumerate}
        \item{I installed "Proxmox VE (Terminal UI)", as I didn't see the need for a
        GUI.} 
        \item{Everything can be done from the Proxmox Web UI, from configuring
        storage to launching VMs and containers.}
        \item{I had quite a bit of trouble with the initial installation. I'd never
            touched an enterprise server like this before.}
    \end{enumerate}
\end{frame}
%%%%%%%%%%%%%%%%%%%%%%%%%%%%%%%%%%%%%%%%%%%%%%%

%%%%%%%%%%%%%% FRAME 4 (Picture) %%%%%%%%%%%%%%%%%%%%
\begin{frame}{}
    \begin{figure}
        \centering
        \includegraphics[width=0.5\linewidth]{pve_setup.jpeg}
        \caption{Photo of the server during setup}
        \label{fig:setup-pic}
    \end{figure}
\end{frame}
%%%%%%%%%%%%%%%%%%%%%%%%%%%%%%%%%%%%%%%%%%%%%%%

%%%%%%%%%%%%%% FRAME: Deciding on storage %%%%%%%%%%%%%%%%%%%%
\begin{frame}{Deciding on the Type of Storage to Use}
    I had a hard time deciding what type of storage to use. 
        I tried making partitions using `ext4` type filesystems, but found they didn't support VM snapshots either.
    \begin{enumerate}
        \item{I initially started using LVM. But this did not support snapshots or
            data sharing.}
        \item{I looked into Ceph, since it has native support in Proxmox, but setup
            looked a little too complex for the timeline of this project.}
        \item{I decided to use ZFS. Also has native support, with feature support 
            (snapshots/sharing).}
    \end{enumerate}
\end{frame}
%%%%%%%%%%%%%%%%%%%%%%%%%%%%%%%%%%%%%%%%%%%%%%%

%%%%%%%%%%%%%% FRAME: USING ZFS %%%%%%%%%%%%%%%%%%%%
\begin{frame}{Creating ZFS for Storage}
    I wiped the two disks I was using for `ext4` filesystems and converted them
    to a `zfs` pool.  

    Then I used the Proxmox Web UI to provision the `zfs` pool for virtual
    machines.  
 
\verb+

wipefs -a /dev/sdb /dev/sdc;  
 
zpool create vmdata /dev/sdb/ /dev/sdc;  

 

 
\+
\end{frame}
%%%%%%%%%%%%%%%%%%%%%%%%%%%%%%%%%%%%%%%%%%%%%%%


%%%%%%%%%%%%%% FRAME Central VM Setup %%%%%%%%%%%%%%%%%%%%
\begin{frame}{Create a Centralized VM}
    \begin{enumerate}
        \item{Creating a centralized VM to host the monitoring tools.} 
        \item{All other nodes will send metrics and logs to this VM.}
        \item{I gave this VM 2 CPUs, 6GB RAM, and 80GB disk space.}
            % TODO: Get a drawing (excalidraw etc) to show diagram of this
    \end{enumerate}
\end{frame}
%%%%%%%%%%%%%%%%%%%%%%%%%%%%%%%%%%%%%%%%%%%%%%%


%%%%%%%%%%%%%% EMPTY FRAME %%%%%%%%%%%%%%%%%%%%
\begin{frame}{}
    \begin{enumerate}
        \item{} 
        \item{}
        \item{}
    \end{enumerate}
\end{frame}
%%%%%%%%%%%%%%%%%%%%%%%%%%%%%%%%%%%%%%%%%%%%%%%

\begin{frame}{References}
    \bibliographystyle{apalike}
    \begin{enumerate}
        \item Proxmox Documentation
        \item 
    \end{enumerate}
    \bibliography{bib}
\end{frame}

\end{document}
